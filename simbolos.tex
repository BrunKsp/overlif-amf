% see https://www.overleaf.com/learn/latex/Nomenclatures

\renewcommand{\nomname}{Lista de Símbolos}
%% This will add the units
%----------------------------------------------
\newcommand{\nomunit}[1]{%
\renewcommand{\nomentryend}{\hspace*{\fill}#1}}
%----------------------------------------------
\renewcommand\nomgroup[1]{%
   \item[\bfseries
   \ifstrequal{#1}{C}{Constantes}{%
   \ifstrequal{#1}{V}{Variáveis}{}}%
]}

%\nomenclature[C]{\(c\)}{Velocidade da luz no vácuo}
%\nomenclature[C]{\(h\)}{Constante de Planck}
%\nomenclature[C]{\(G\)}{Constante Gravitacional}
% Se quiser obter o valor externamente:
\nomenclature[C]{\(c\)}{\href{https://physics.nist.gov/cgi-bin/cuu/Value?c}{Velocidade da luz no vácuo}
\nomunit{\SI{299792458}{\meter\per\second}}}
\nomenclature[C]{\(h\)}{\href{https://physics.nist.gov/cgi-bin/cuu/Value?h}{Constante de Planck}
\nomunit{\SI[group-digits=false]{6.62607015e-34}{\joule\per\hertz}}}
\nomenclature[C]{\(G\)}{\href{https://physics.nist.gov/cgi-bin/cuu/Value?bg} {Constante Gravitacional} 
\nomunit{\SI[group-digits=false]{6.67430e-11}{\meter\cubed\per\kilogram\per\second\squared}}}
%\nomenclature[C]{\(\mathbb{R}\)}{Números reais}
%\nomenclature[C]{\(\mathbb{C}\)}{Números complexos}
\nomenclature[V]{\(\mathbb{H}\)}{Matriz Hermitiana}
\nomenclature[V]{\(V\)}{Volume}
\nomenclature[V]{\(\alpha\)}{Coeficiente de absorção sonora}
\nomenclature[V]{\(p\)}{Pressão sonora}
%\nomenclature[S]{ABNT}{Associação Brasileira de Normas Técnicas}
%\nomenclature[S]{UFRJ}{Universidade Federal do Rio de Janeiro}
%\nomenclature[S]{DENATRAN}{Departamento Nacional de Trânsito}
%\nomenclature[A]{FVM}{Faça você mesmo(a)}
%\nomenclature[A]{IOC}{Internet das coisas}

