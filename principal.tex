%%%%%%%%%%%%%%%%%%%%%%%%%%%%%%%%%%%%%%%%%%%%%%%%%%%%%%%%%%%
%%% SELECIONE UM DOS MODELOS DE CITAÇÂO, 
%%% REMOVENDO O COMENTÁRIO (%) DA RESPECTIVA LINHA
%%%%%%%%%%%%%%%%%%%%%%%%%%%%%%%%%%%%%%%%%%%%%%%%%%%%%%%%%%%
%\documentclass[tcc]{tcc-poli}   %% citações em formato numérico
\documentclass[tcc,alf]{tcc-poli}  %% citações em formato alfabético

%%%%%%%%%%%%%%%%%%%%%%%%%%%%%%%%%%%%%%%%%%%%%%%%%%%%%%%%%%%
% NÂO ALTERE AS CONFIGURAÇÕES E PACOTES ATÈ O TRECHO DO 
% CÓDIGO/TEXTO INDICADO PARA TAL
%%%%%%%%%%%%%%%%%%%%%%%%%%%%%%%%%%%%%%%%%%%%%%%%%%%%%%%%%%%

%% Pacotes
\usepackage[utf8]{inputenc}
\usepackage{float} 
\usepackage{appendix}
\usepackage{pdflscape}
\usepackage{multirow}
\usepackage{longtable}
\usepackage{graphics}
\usepackage{times}
\usepackage{enumerate}
\usepackage[hidelinks]{hyperref}
\usepackage{epstopdf}
\usepackage{lipsum}
\usepackage{amssymb}
\usepackage{fancyvrb} 
\usepackage{etoolbox}
\usepackage{siunitx}
\usepackage{listings}
\usepackage{caption}
\usepackage{subcaption}
\usepackage[normalem]{ulem}
\useunder{\uline}{\ul}{}

\usepackage{graphicx} %package to manage images
\graphicspath{ {./Figuras/} } %% pasta para organizar as figuras

%%% AJUSTE DOS TAMANHOS E ESTILOS DOS CABECALHOS
%%% NÂO MODIFICAR
\usepackage[big]{titlesec}
\titleformat{\chapter}
{\normalfont\LARGE\bfseries}{\thechapter}{1em}{}
\titleformat{\section}
{\normalfont\Large\bfseries}{\thesection}{1em}{}
\titleformat{\subsection}
{\normalfont\large\bfseries}{\thesubsection}{1em}{}
\titleformat{\subsubsection}
{\normalfont\normalsize\bfseries}{\thesubsubsection}{1em}{}
\titleformat{\paragraph}[runin]
{\normalfont\normalsize\bfseries}{\theparagraph}{1em}{}
\titleformat{\subparagraph}[runin]
{\normalfont\normalsize\bfseries}{\thesubparagraph}{1em}{}
% \titleformat*{\chapter}{\LARGE\bfseries}
% \titleformat*{\section}{\LARGE\bfseries}
% \titleformat*{\subsection}{\Large\bfseries}
% \titleformat*{\subsubsection}{\large\bfseries}
% \titleformat*{\paragraph}{\large\bfseries}
% \titleformat*{\subparagraph}{\large\bfseries}

% para lista de simbolos
\usepackage[intoc]{nomencl}
\makenomenclature

% para lista de abreviaturas
 \usepackage[nomain,acronym,xindy,toc,noredefwarn]{glossaries}
 \makeglossaries

% ABNTEX ...
%% Em próximas versões tentar usar o BIBLATEX em susbstituição do abntex2cite
%\usepackage[
%style = abnt, % Sistema alfabético
% style = abnt-numeric, % Sistema numérico
% style = abnt-ibid, % Notas de referência
%]{biblatex}
%% NÃO alterar os comandos abaixo
\ifx\AbntTexType\StringNum
  \usepackage[num]{abntex2cite}
  \citebrackets[] 
\else
  \usepackage[alf]{abntex2cite}
\fi

%%%%%%%%%%%%%%%%%%%%%%%%%%%%%%%%%%%%%%%%%%%%%%%%%%%%%%%%%%%%
%% A PARTIR DAQUI FAÇA AS ALTERAÇÔES NOS RESPECTIVOS CAMPOS
%%%%%%%%%%%%%%%%%%%%%%%%%%%%%%%%%%%%%%%%%%%%%%%%%%%%%%%%%%%%
%
%% Definicoes
\def\dnl{\mbox{DNL }}
\def\DNL{\mbox{DNL }}

\begin{document}

  %% preencha aqui o Título da Dissertação 
  \title{eu sou o maiorarl}
  
  %% Título em Ingles
  \foreigntitle{Template for Graduation Project - POLI/UFRJ}
  %% Autor da dissertação
   \author{Nome do(a) Autor(a)} {Sobrenome}
  %% Orientador(es) - adicionar outras linhas caso haja mais orientadores
  \advisor{Prof.}{Nome}{e Sobrenome do(a) Orientador(a)}{D.Sc.} % 
  %\advisor{Prof.}{Nome}{e Sobrenome do(a) Orientador(a) 2}{D.Sc.}  % se houver 2 orientadores
  %\advisor{Prof.}{Nome}{Sobrenome}{D.Sc.}
  
  %% Membros da banca 
  \examiner{Prof.}{Nome completo do membro 1}{D.Sc.}
  \examiner{Prof.}{Nome completo do membro 2}{Ph.D.}
  \examiner{Prof.}{Nome completo do membro 3}{Ph.D.}
  \examiner{Prof.}{Nome completo do membro 4}{D.Sc.}
  % \examiner{Prof.}{Ana Paula Gama}{D.Sc.}

  %% selecionar um Curso (remover comentário do respectivo curso):
  %\engcourse{AMB} % Engenharia Ambiental
  \engcourse{ECV} % Engenharia Civil
  %\engcourse{ECI} % Engenharia de Computação e Informação
  %\engcourse{ECA} % Engenharia de Controle e Automação
  %\engcourse{ELT} % Engenharia Eletrônica e de Computação
  %\engcourse{ELE} % Engenharia Elétrica
  %\engcourse{MAT} % Engeharia de Materiais
  %\engcourse{MET} % Engenharia Metalúrgica
  %\engcourse{MEC} % Engenharia Mecânica
  %\engcourse{NAV} % Engenharia Naval e Oceânica
  %\engcourse{NUC} % Engenharia Nuclear
  %\engcourse{PET} % Engenharia de Petróleo
  %\engcourse{PRO} % Engenharia de Produção
  
  %% Mês e Ano da defesa (não precisa alterar)
  \date{\the\month{}}{\the\year{}}

  %% Palavras-chave em ingles ficam no absract.
  \keyword{Palavra-chave1}
  \keyword{Palavra-chave2}
  \keyword{Palavra-chave3}
  \keyword{Palavra-chave4}

  %%%%%%%%%%%%%%%%%%%%%%%%%%%%%%%%%%%%%%%%%%
  %%  PARTE PRÉ-TEXTUAL. (NÃO MODIFICAR)
  \maketitle
  \frontmatter
  
   \mbox{}\vfill
\begin{flushright}
\begin{minipage}{0.5\textwidth}
\textit{Dedico essa dissertação a Deus, que me sustentou na caminhada, aos meus pais e aos amigos que me incentivam a voar!}\par
(Opcional)
\end{minipage}
\end{flushright}
 %% OPCIONAL 
  Agradeço a \lipsum[2]

 




  \begin{abstract}
\noindent
%% substituir pelo texto do resumo em parágrafo único
\lipsum[1]

\end{abstract}

\keywords{}






  \begin{foreignabstract}
%Texto do resumo em inglês
\lipsum[3]
\end{foreignabstract}

\vspace*{0.5cm}
\noindent 
\textit{Keywords:} keyword 1, keyword 2, keyword 3, keyword 4. \par
%\keywords[english]{one, two, three, four}  
  \tableofcontents 
  \listoffigures
  \listoftables
  \listofquadros
  % see https://www.overleaf.com/learn/latex/Nomenclatures

\renewcommand{\nomname}{Lista de Símbolos}
%% This will add the units
%----------------------------------------------
\newcommand{\nomunit}[1]{%
\renewcommand{\nomentryend}{\hspace*{\fill}#1}}
%----------------------------------------------
\renewcommand\nomgroup[1]{%
   \item[\bfseries
   \ifstrequal{#1}{C}{Constantes}{%
   \ifstrequal{#1}{V}{Variáveis}{}}%
]}

%\nomenclature[C]{\(c\)}{Velocidade da luz no vácuo}
%\nomenclature[C]{\(h\)}{Constante de Planck}
%\nomenclature[C]{\(G\)}{Constante Gravitacional}
% Se quiser obter o valor externamente:
\nomenclature[C]{\(c\)}{\href{https://physics.nist.gov/cgi-bin/cuu/Value?c}{Velocidade da luz no vácuo}
\nomunit{\SI{299792458}{\meter\per\second}}}
\nomenclature[C]{\(h\)}{\href{https://physics.nist.gov/cgi-bin/cuu/Value?h}{Constante de Planck}
\nomunit{\SI[group-digits=false]{6.62607015e-34}{\joule\per\hertz}}}
\nomenclature[C]{\(G\)}{\href{https://physics.nist.gov/cgi-bin/cuu/Value?bg} {Constante Gravitacional} 
\nomunit{\SI[group-digits=false]{6.67430e-11}{\meter\cubed\per\kilogram\per\second\squared}}}
%\nomenclature[C]{\(\mathbb{R}\)}{Números reais}
%\nomenclature[C]{\(\mathbb{C}\)}{Números complexos}
\nomenclature[V]{\(\mathbb{H}\)}{Matriz Hermitiana}
\nomenclature[V]{\(V\)}{Volume}
\nomenclature[V]{\(\alpha\)}{Coeficiente de absorção sonora}
\nomenclature[V]{\(p\)}{Pressão sonora}
%\nomenclature[S]{ABNT}{Associação Brasileira de Normas Técnicas}
%\nomenclature[S]{UFRJ}{Universidade Federal do Rio de Janeiro}
%\nomenclature[S]{DENATRAN}{Departamento Nacional de Trânsito}
%\nomenclature[A]{FVM}{Faça você mesmo(a)}
%\nomenclature[A]{IOC}{Internet das coisas}

 
  \printnomenclature
  %usado com glossaries:
 % exemplos de abreviaturas
\newacronym{IA}{IA}{Inteligência Artificial}         
\newacronym{RBC}{RBC}{Raciocínio Baseado em Casos}
\newacronym{IDC}{IDC}{Internet das coisas}  
%exemplos de siglas
\newacronym{ABNT}{ABNT}{Associação Brasileira de Normas Técnicas}  
\newacronym{UFRJ}{UFRJ}{Universidade Federal do Rio de Janeiro}  




  \cleardoublepage
  \printglossary[title=Lista de Siglas e Abreviaturas, type=\acronymtype]
  \mainmatter
  %%%%%%%%%%%%%%%%%%%%%%%%%%%%%%%%%%%%%%%%%%
  %% PARTE TEXTUAL
  
  %% Capítulos 
  \chapter{Introdução}

Apresentar uma descrição muito ampla e generalista do contexto no qual a pesquisa está inserida, o problema em questão e a motivação para realizar essa pesquisa

%% substituir por texto 
\lipsum[1]

\section{Objetivo}
\subsection{Objetivo Geral}

Apresentar CLARAMENTE o objetivo do trabalho, ou seja, O QUE está sendo feito. Os objetivos, geralmente, são verbos como analisar, desenvolver, comparar, propor, elaborar, identificar etc.

\subsection{Objetivo Específico}

Dentro desse objetivo geral, encontram-se como metas específicas:
\begin{itemize}
    \item Identificar algo; 
  
    \item Comparar alguma coisa;
   
   \item Fazer um diagnóstico ...;
   
   \item Verificar e analisar ...;
   
    \item Quantificar ...
\end{itemize}

\section{Metodologia}

Apresentar COMO o seu problema será ``resolvido", ou seja, quais são as etapas necessárias para atingir o(s) objetivo(s).

Não explique aqui O QUE é uma metologia ou os tipos de metologias que existem. 
Explique A SUA metodologia para alcançar o objetivo desejado.



\section{Estrutura do Trabalho}

Colocar aqui a estrutura sucintamente. Esse trabalho está organizado ... No capítulo 2 é apresentado isso . . Tal coisa é apresentada no cap. 3 ...


  \chapter{Revisão Bibliográfica}

Este capítulo apresenta uma revisão bibliográfica sobre o estado da arte e a definição dos conceitos e dos aspectos necessários para a compreensão do trabalho.

\section{Conceituação Básica sobre Algo}

Nesse capítulo estará toda a fundamentação teórica no estado atual da arte para que os leitores compreendam as etapas metodológicas que serão descritas posteriormente, bem como para a apresentação dos resultados. Nele devem ser colocadas as referências bibliográficas.  

\subsection{título de uma subseção}

Inclua subseções conforme convier.

\section{Como incluir referências bibliográficas}

Esse modelo de Projeto de Graduação utiliza o pacote ABNTEX2CITE para citações, devido ao uso das normas ABNT.

Para saber como fazer as citações e as possibilidades que existem, consulte os sites abaixo, principalmente o manual no primeiro link:
\begin{itemize}
\item \textcolor{red}{\href{http://mirrors.ctan.org/macros/latex/contrib/abntex2/doc/abntex2cite-alf.pdf}{http://mirrors.ctan.org/macros/latex/contrib/abntex2/doc/abntex2cite-alf.pdf}}
    \item \href{https://www.abntex.net.br}{https://www.abntex.net.br}
    \item \href{https://github.com/abntex/abntex2}{https://github.com/abntex/abntex2}
\end{itemize}

Para incluir uma referência bibliográfica, primeiro adione-a ao arquivo referencias.bib, de acordo o tipo, ou seja, tese, dissertação (msthesis), livro (book), manual (manual), artigo de revista (article), artigo de congressos (inproceedings). Em seguinta adicione o seu texto acrecentando a chamada para o local onde deseja incluir a referência \cite{OMS2011} ou diversas juntas: \cite{NBR12859,who2018}.


\subsection{Exemplos}

\lipsum[2] \cite{infraero}.

\lipsum[2] \cite{BAYER2009}

\lipsum[2] \cite{googleearth}.

\lipsum[2] \cite{OMS2011}.

\lipsum[2] \cite{NBR12859}.

\subsection{Como incluir Figuras}

Para incluir figuras, primeiro faça o upload do arquivo no formato desejado (JPG, EPS, PNG etc) na pasta de figuras ou em qualquer pasta que desejar. Em seguida, use o texto abaixo como exemplo para incluir figuras, substituindo o nome do arquivo e o identificador dessa figura no texto através do ''label´´. Esse deve ser um identificador único o longo de todo o texto. Sugere-se que o label das figuras comece com ``fig:", dessa forma fica mais fácil selecionar figuras, tabelas e equações dentre a lista de opções que aparecem para inserir os identificadores ao longo do texto.
Para fazer referência a uma figura no texto basta utilizar o comando ``ref", como mostrado na Figura \ref{fig:conceitos-da-onda}
\begin{figure}[H]
	\centering
	\includegraphics[width=11cm]{Figuras/capitulo2/onda_sonora.jpg}
	\caption{Amplitude da onda sonora.}
	Fonte: Adaptado de ... .
    \label{fig:conceitos-da-onda}
\end{figure}

\subsection{Subfiguras}

Para inserir diversas imagens dentro de uma unica figura use o exemplo abaixo. 
Para fazer referencia à figura com todas as imagens use: Fig.~\ref{fig:exemplos-de-subfiguras}. Para fazer referência às subfiguras use os respectivos labels Fig.~\ref{fig:exemplo-subfig1} e Fig.~\ref{fig:exemplo-subfig2}.

\begin{figure}[H]
    \centering
    \begin{subfigure}{\textwidth}
       \centering
       \includegraphics[width=.5\textwidth]{Figuras/capitulo2/onda_sonora.jpg}
       \caption{Primeira imagem.}
       \label{fig:exemplo-subfig1}
    \end{subfigure}
    \begin{subfigure}{\textwidth}
       \centering
       \includegraphics[width=.7\textwidth]{Figuras/capitulo2/graficosel.jpg}
       \caption{Segunda imagem.}
       \label{fig:exemplo-subfig2}
    \end{subfigure}
  \caption{Legenda da figura completa (com as diversas imagens).}
  \label{fig:exemplos-de-subfiguras}
\end{figure}

Veja mais sobre subfiguras no link: \url{https://latex-tutorial.com/subfigure-latex/}

\subsection{Como incluir Equações}

\begin{equation}
    \texttt{NPS} = 10 \log_{10} \left (\frac{p^2(t)}{p_0^2} \right ),
\end{equation}
onde ${p(t)}$ corresponde a uma variação de pressão sonora em relação a um valor constante ($p_0$), que representa o limiar inferior da audição, definido como $20\mu Pa$. 

O nível correspondente ao período de 15 horas do dia, entre 7h e 22h, é conhecido como \textit{Nível Equivalente Dia ($L_D$)} e pode ser obtido pela Equação \ref{L_D}:
    \begin{equation}
    L_D =  10 \log_{10}\left( \frac{1}{N} \sum_{n=1}^{N} 10^{\frac{L_{Aeq,t}[n]}{10}} \right) \mbox{dB},
    \label{L_D}
    \end{equation}	
considerando amostras do nível $N$ obtidas durante o período $ T $ de tempo. Da mesma forma, o período noturno $L_N$, é calculado pela Equação \ref{L_D}, exceto por conter as amostras de nível tomadas das 22h às 7h do dia seguinte.    

 Nível Médio de Ruído Dia-Noite (\textit{Day-Night Average Noise Level}), \dnl ou L$_ {DN} $), é dado pela média ponderada da energia sonora durante 24 horas. O \dnl é definido pela Equação \ref{Ldn} ~\cite{SILVA2016}, considerando o tempo total de 24 horas e aplicando uma penalidade de 10 dB ao ruído medido durante o período noturno:

\begin{equation}
	\dnl = 10 \log_{10} \left\{ \frac{1}{24} \left[ 15 \cdot 10^{\frac{L_D}{10}} + 9 \cdot 10^{\frac{(L_N + 10)}{10}} \right] \right\}.
	\label{Ldn}
\end{equation}

\section{As Tabelas}

\begin{table}[H]
\centering
\caption{Limites de níveis de pressão sonora de acordo com a área habitada e período}
\resizebox{\textwidth}{!}{%
\begin{tabular}{|l|c|c|}
\hline
\multirow{2}{*}{\textbf{Tipos de Áreas Habitadas}} & \multicolumn{2}{c|}{\textbf{\begin{tabular}[c]{@{}c@{}}$RL_{Aeq}$\\ Limites dos níveis de\\ pressão sonora (dB)\end{tabular}}} \\ \cline{2-3} 
 & \begin{tabular}[c]{@{}c@{}}Período Diurno\end{tabular} & \begin{tabular}[c]{@{}c@{}}Período Noturno\end{tabular} \\ \hline
Residencial rural & 40 & 35 \\ \hline
Estritamente residencial urbana ou de hospitais ou de escolas & 50 & 45 \\ \hline
Mista predominantemente residencial & 55 & 50 \\ \hline
Mista com predominância de atividades comerciais e/ou administrativa & 60 & 55 \\ \hline
Mista com predominância de atividades culturais, lazer e turismo & 65 & 55 \\ \hline
Predominantemente industrial & 70 & 60 \\ \hline
\end{tabular}%
}
(Modificado de \cite{WYLE2011}).
\label{limitesNBR10151}
\end{table}


Um exemplo é mostrado na tabela \ref{tab:tabela_hap}, onde são apresentados os resultados.
\begin{table}[H] 
\centering
\caption{\%HAP segundo diversos Autores e agências ambientais}
\label{tab:tabela_hap}
\resizebox{\textwidth}{!}{%
\begin{tabular}{
*8{|>{\centering\arraybackslash}m{1.7 cm}}
|}
\hline
{DNL (dB)} & {Schultz} & {Fidell, Schultz e Barber} & {Miedema e Vos} & {EPA} & {NRC} & {NRC aproximado} & {OECD} \\ \hline
50 & 1,16 & 5,69 & 5,28 & 7,2 & 2,26 & 2,26 & 0 \\ \hline
55 & 3,82 & 8,27 & 11,04 & 16,2 & 4,57 & 4,52 & 10 \\ \hline
60 & 8,36 & 12,65 & 18,56 & 25,2 & 8,67 & 8,58 & 20 \\ \hline
65 & 15,11 & 18,82 & 27,76 & 34,2 & 15,17 & 15,45 & 30 \\ \hline
70 & 21,45 & 26,8 & 38,51 & 43,2 & 24,49 & 24,72 & 40 \\ \hline
75 & 36,72 & 36,58 & 50,71 & 52,2 & 36,86 & 37,08 & 50 \\ \hline
80 & 52,26 & 48,16 & 64,27 & 61,2 & - & 51,92 & 60 \\ \hline
85 & 71,45 & 61,54 & 79,07 & 70,2 & - & 67,49 & 70 \\ \hline
90 & 94,62 & 76,71 & 95,01 & 79,2 & - & 80,99 & 80 \\ \hline
95 & 122,13 & 93,69 & 111,99 & 88,2 & - & 89,09 & 90 \\ \hline
\end{tabular}%
}
Fonte: Adaptado de ... .
\end{table}

\section{Itens}

Lista sem deslocamento

\begin{itemize}
        \item {\bf Impactos Econômicos diretos}: através da geração de empregos, atração de investimento de capital e recolhimento de impostos;
        \item {\bf Impactos econômicos indiretos}: estímulo e desenvolvimento do setor de turismo, investimento em infraestrutura, atratividade na instalação de filiais de grandes empresas e melhoria na estrutura urbana;
        \item {\bf Impactos ambientais}: tratamento de resíduos, ruído aeroportuário e de transportes terrestres, eficiência energética, controle de poluição e controle de espécies de aves e da vegetação.
\end{itemize}

e sem deslocamento
\begin{quote}
   \begin{itemize}
        \item {\bf Impactos Econômicos diretos}: através da geração de empregos, atração de investimento de capital e recolhimento de impostos;
        \item {\bf Impactos econômicos indiretos}: estímulo e desenvolvimento do setor de turismo, investimento em infraestrutura, atratividade na instalação de filiais de grandes empresas e melhoria na estrutura urbana;
        \item {\bf Impactos ambientais}: tratamento de resíduos, ruído aeroportuário e de transportes terrestres, eficiência energética, controle de poluição e controle de espécies de aves e da vegetação.
    \end{itemize}
\end{quote}


  \chapter{Título do capítulo 3}

Este capítulo, em geral, explica os procedimentos da metodologia.

\section{Breve histórico}

%substituir
\lipsum[1]

\section{Uso de Siglas e Abreviaturas no texto}
Para usar as siglas e abreviações definidas no arquivo abrev.txt, podem ser usadas de 3 formas:
Se quiser escrever ``Racícionio Baseado em Casos (RBC)'' de forma completa, ou seja, com nome completo e a sigla entre parênteses use o comando $\backslash$acrfull\{RBC\}:
Usando o \acrfull{RBC} é possível ...
Se desejar usar apenas a sigla escreva $\backslash$acrshort\{RBC\}:
O \acrshort{RBC} é muito útil ...

Se for usar apenas o texto, sem a sigla, utilize o comando $\backslash$acrlong\{RBC\}

O \acrlong{RBC} é muito usado em ...

Outro exemplo é o uso de \acrfull{IA}. 


Somente as siglas e abreviaturas cadastradas no arquivo abrev.txt \textbf{E UTILIZADAS NO TEXTO} serão inseridas na Lista de Siglas, com os respectivos links para o local onde a sigla foi usada pela primeira vez.

\section{Símbolos e Variáveis}

Os símbolos e variáveis usadas ao longo do texto podem ser incluídas em uma Lista de Símbolos na parte pré-textual.
Para isso, crie primeiro a lista de símbolos, conforme descrito no arquivo simbolos.tex. Em seguida use as definições no texto 

% \begin{equation} \label{eu_eqn}
% \begin{split}
% Lending_{it} = {\abscoef} + {\tempo}\
% \end{split}
% \end{equation}

\section{Quadros ...}

Somente no Brasil são usados ``Quadros'' na ABNT.
Veja como colocar um quadro: 
O Quadro~\ref{qd:Teste} apresenta um exemplo de um quadro no formato de tabela.
\begin{quadro}[h]
  \caption{Exemplo de um quadro formatado como tabela (o caption dos quadros vem antes do quadro)}
\begin{center}
  \begin{tabular}{|c|c|} \hline
     Informação A & Alguma coisa relativa \\ \hline
     Informação B & Outra coisa \\ \hline
  \end{tabular}
\end{center}
\label{qd:Teste}
\end{quadro}

Outro exemplo.

\begin{quadro}[h]
\caption{Legenda de um quadro qualquer}
\label{quad:primeiro_quadro}
\centering
\begin{tabular}{|lllll|}
\cline{1-5}
A& B & C& D &E \\ \cline{1-5}
\multirow{3}{*}{1} & 2 & 3& 4& 5 \\
& 2 & 3& 4& 5 \\
& 2 & 3& 4& 5 \\
\cline{1-5}
\end{tabular}
\end{quadro}

Mais um exemplo:

\begin{quadro}[htb]
\caption{\label{quadro_exemplo}Exemplo de quadro}
\begin{center}
\begin{tabular}{|c|c|c|c|}
	\hline
	\textbf{Pessoa} & \textbf{Idade} & \textbf{Peso} & \textbf{Altura} \\ \hline
	Marcos & 26    & 68   & 178    \\ \hline
	Ivone  & 22    & 57   & 162    \\ \hline
	...    & ...   & ...  & ...    \\ \hline
	Sueli  & 40    & 65   & 153    \\ \hline
\end{tabular}\par 
Fonte: o Autor.
\end{center}
\end{quadro}

Este parágrafo apresenta como referenciar o quadro no texto, requisito
obrigatório da ABNT. 
Primeira opção, utilizando \texttt{autoref}: Ver o \autoref{quadro_exemplo}. 
Segunda opção, utilizando  \texttt{ref}: Ver o Quadro \ref{quadro_exemplo}.
  \chapter{Avaliação dos Resultados}

Neste capítulo são avaliados os resultados do trabalho ...

\lipsum[3]

  \chapter{Conclusões}

Este trabalho buscou ... 

\lipsum[3]
  
  \backmatter

  %%%%%%%%%%%%%%%%%%%%%%%%%%%%%%%%%%%%%%%%%%%%
  %% REFERÊNCIAS BIBLIOGRÁFICAS: 
  %% veja como fazer as citações em 
  %% http://mirrors.ctan.org/macros/latex/contrib/abntex2/doc/abntex2cite.pdf
  %%%%%%%%%%%%%%%%%%%%%%%%%%%%%%%%%%%%%%%%%%%%
  %% Padroes ABNT  (NÂO MODIFICAR!)
  \ifx\AbntTexType\StringNum
    \bibliographystyle{abntex2-num}
  \else
    \bibliographystyle{abntex2-alf}
  \fi
  %%%%%%%%%%%%%%%%%%%%%%%%%%%%%%%%%%%%%%%%%%%%

  %% Arquivo com os dados das referências
  \bibliography{referencias}

  %% Apêndices e anexos
  \appendix
  \chapter{Título do Anexo 1}
\label{apendiceA}
Conteúdo do Anexo \ref{apendiceA}


\begin{landscape}
%listagem ensino
\footnotesize
\begin{longtable}[c]{|c|l|l|l|c|c|c|} 
\caption{Identificação das Unidades de Ensino}
\label{siglaescola}
\\ \hline 
  \textbf{Sigla} &
  \textbf{Nome} &
  \textbf{Tipo} &
  \textbf{Endereço} &
  \textbf{Município} &
  \textbf{Latitude} &
  \textbf{Longitude} \\ \hline
\endfirsthead
%
\multicolumn{7}{c}%
{{\bfseries Tabela \thetable\ continuação da página anterior}} \\
\hline
\textbf{Sigla} &
  \textbf{Nome} &
  \textbf{Tipo} &
  \textbf{Endereço} &
  \textbf{Município} &
  \textbf{Latitude} &
  \textbf{Longitude} \\ \hline
\endhead
%
E-FM 1 &
  \begin{tabular}[c]{@{}l@{}}EEEFM Belmiro \\ Teixeira Pimenta\end{tabular} &
  \begin{tabular}[c]{@{}l@{}}Escola Estadual de Ensino \\ Fundamental e Médio\end{tabular} &
  \begin{tabular}[c]{@{}l@{}}Rua dos Perdizes, 5, \\ Eurico Salles\end{tabular} &
  Serra &
  -20,237849 &
  -40,276399 \\ \hline
E-FM 2 &
  EEEFM Rômulo Castello &
  \begin{tabular}[c]{@{}l@{}}Escola Estadual de Ensino \\ Fundamental e Médio\end{tabular} &
  \begin{tabular}[c]{@{}l@{}}Rua Independência, s/n, \\ Rosário de Fátima\end{tabular} &
  Serra &
  -20,231038 &
  -40,273138 \\ \hline
E-FM 3 &
  \begin{tabular}[c]{@{}l@{}}EEEFM Aflordízio \\ Carvalho da Silva\end{tabular} &
  \begin{tabular}[c]{@{}l@{}}Escola Estadual de Ensino \\ Fundamental e Médio\end{tabular} &
  \begin{tabular}[c]{@{}l@{}}Rua Engenheiro Rubens\\Bley, 100, Jardim da Penha\end{tabular} &
  Vitória &
  -20,295366 &
  -40,311448 \\ \hline
E-FM 4 &
  \begin{tabular}[c]{@{}l@{}}EEEFM Professora \\ Juraci Machado\end{tabular} &
  \begin{tabular}[c]{@{}l@{}}Escola Estadual de Ensino \\ Fundamental e Médio\end{tabular} &
  Av. Santarém, s/n - Barcelona &
  Serra &
  -20,174800 &
  -40,243826 \\ \hline
E-FM 5 &
  EEEFM Maria Penedo &
  \begin{tabular}[c]{@{}l@{}}Escola Estadual de Ensino \\ Fundamental e Médio\end{tabular} &
  Av. Guarapari, s/n - Valparaíso &
  Serra &
  -20,198483 &
  -40,261433 \\ \hline
E-FM 6 &
  \begin{tabular}[c]{@{}l@{}}EEEFM Aristóbulo \\ Barbosa Leão\end{tabular} &
  \begin{tabular}[c]{@{}l@{}}Escola Estadual de Ensino \\ Fundamental e Médio\end{tabular} &
  \begin{tabular}[c]{@{}l@{}}Av. Desembargador Mário da Silva\\ Nunes, 1000 - Jardim Limoeiro\end{tabular} &
  Serra &
  -20,209231 &
  -40,262674 \\ \hline
E-FM 7 &
  EEEFM Almirante Barroso &
  \begin{tabular}[c]{@{}l@{}}Escola Estadual de Ensino \\ Fundamental e Médio\end{tabular} &
  \begin{tabular}[c]{@{}l@{}}Rua do Almirante, s/n \\- Goiabeiras\end{tabular} &
  Vitória &
  -20,266991 &
  -40,299516 \\ \hline
E-FM 8 &
  EEEFM Maria Ortiz &
  \begin{tabular}[c]{@{}l@{}}Escola Estadual de Ensino \\ Fundamental e Médio\end{tabular} &
  \begin{tabular}[c]{@{}l@{}}Rua Francisco de Araújo, s/n - \\ Centro\end{tabular} &
  Vitória &
  -20,321185 &
  -40,340150 \\ \hline
E-FM 9 &
  \begin{tabular}[c]{@{}l@{}}EEEFM Dr. Francisco \\ Freitas Lima\end{tabular} &
  \begin{tabular}[c]{@{}l@{}}Escola Estadual de Ensino \\ Fundamental e Médio\end{tabular} &
  \begin{tabular}[c]{@{}l@{}}Rua Antônio Abraão, 20 -\\ Ilha das Flores\end{tabular} &
  Vila Velha &
  -20,332892 &
  -40,334423 \\ \hline
E-FM10 &
  \begin{tabular}[c]{@{}l@{}}EEEFM Agenor de \\ Souza Lé\end{tabular} &
  \begin{tabular}[c]{@{}l@{}}Escola Estadual de Ensino \\ Fundamental e Médio\end{tabular} &
  \begin{tabular}[c]{@{}l@{}}Rua Alan Kardec - S/n - \\ Divino Espírito Santo\end{tabular} &
  Vila Velha &
  -20,346425 &
  -40,298591 \\ \hline
E-FS 1 &
  \begin{tabular}[c]{@{}l@{}}ESESP - Escola de Serviço \\ Público do Espírito Santo\end{tabular} &
  \begin{tabular}[c]{@{}l@{}}Centro de Formação \\ de Servidores\end{tabular} &
  \begin{tabular}[c]{@{}l@{}}Rua Francisco Fundão, 155 - \\ Morada de Camburí\end{tabular} &
  Vitória &
  -20,272087 &
  -40,294358 \\ \hline
E-M 1 &
  EEEM Amulpho Mattos &
  \begin{tabular}[c]{@{}l@{}}Escola Estadual \\ de Ensino Médio\end{tabular} &
  \begin{tabular}[c]{@{}l@{}}Rua Presidente Nereu Ramos,\\172, República\end{tabular} &
  Vitória &
  -20,271889 &
  -40,294160 \\ \hline
E-M2 &
  \begin{tabular}[c]{@{}l@{}}EEEM Prof. Renato José \\ da Costa Pacheco\end{tabular} &
  \begin{tabular}[c]{@{}l@{}}Escola Estadual \\ de Ensino Médio\end{tabular} &
  \begin{tabular}[c]{@{}l@{}}Av. Engenheiro Charles Bitran,\\ 251 - Jardim Camburi\end{tabular} &
  Vitória &
  -20,246631 &
  -40,271606 \\ \hline
E-M3 &
  EEEM Irmã Maria Horta &
  \begin{tabular}[c]{@{}l@{}}Escola Estadual \\ de Ensino Médio\end{tabular} &
  \begin{tabular}[c]{@{}l@{}}Rua Aleixo Netto, 1060 - \\ Praia do Canto\end{tabular} &
  Vitória &
  -20,297937 &
  -40,294143 \\ \hline
E-M4 &
  \begin{tabular}[c]{@{}l@{}}EEEM Desembargador\\ Carlos Xavier Paes Barreto\end{tabular} &
  \begin{tabular}[c]{@{}l@{}}Escola Estadual \\ de Ensino Médio\end{tabular} &
  \begin{tabular}[c]{@{}l@{}}Av. Leitão da Silva, s/n - \\ Santa Lucia\end{tabular} &
  Vitória &
  -20,310261 &
  -40,302031 \\ \hline
E-M5 &
  \begin{tabular}[c]{@{}l@{}}CEEMTI Professor \\ Fernando Duarte Rabelo\end{tabular} &
  \begin{tabular}[c]{@{}l@{}}Escola Estadual \\ de Ensino Médio\end{tabular} &
  \begin{tabular}[c]{@{}l@{}}Praça Cristóvão Jaques, 260 - \\ Santa Helena\end{tabular} &
  Vitória &
  -20,310471 &
  -40,294336 \\ \hline
E-M6 &
  Eeem Gomes Cardim &
  \begin{tabular}[c]{@{}l@{}}Escola Estadual \\ de Ensino Médio\end{tabular} &
  \begin{tabular}[c]{@{}l@{}}Rua Wilson de Freitas, s/n - \\ Centro\end{tabular} &
  Vitória &
  -20,317620 &
  -40,330130 \\ \hline
E-M7 &
  \begin{tabular}[c]{@{}l@{}}EEEM Colégio Estadual \\ do Espirito Santo\end{tabular} &
  \begin{tabular}[c]{@{}l@{}}Escola Estadual \\ de Ensino Médio\end{tabular} &
  Av. Vitória, 550 - Forte São João &
  Vitória &
  -20,318402 &
  -40,324518 \\ \hline
E-M8 &
  EEEM Godofredo Scheider &
  \begin{tabular}[c]{@{}l@{}}Escola Estadual \\ de Ensino Médio\end{tabular} &
  \begin{tabular}[c]{@{}l@{}}Rua Bernardo Schineider, 15 - \\ Centro\end{tabular} &
  Vila Velha &
  -20,329850 &
  -40,292653 \\ \hline
E-M9 &
  EEEM Florentino Avidos &
  \begin{tabular}[c]{@{}l@{}}Escola Estadual \\ de Ensino Médio\end{tabular} &
  Av. Vitória Régia, s/n - Ibes &
  Vila Velha &
  -20,353252 &
  -40,317125 \\ \hline
F-I 1 &
  Criarte UFES &
  \begin{tabular}[c]{@{}l@{}}Centro Federal de \\ Educação Infantil\end{tabular} &
  \begin{tabular}[c]{@{}l@{}}Av. Fernando Ferrari, 514, \\ Goiabeiras\end{tabular} &
  Vitória &
  -20,279580 &
  -40,305000 \\ \hline
F-S 1 &
  UFES - Campus Goiabeiras &
  Universidade Federal &
  Av. Fernando Ferrari, 1852, Goiabeiras &
  Vitória &
  -20,275566 &
  -40,301991 \\ \hline
F-S 2 &
  UFES - Campus Maruípe &
  Universidade Federal &
 Av. Marechal Campos, 1468, Bonfim &
  Vitória &
  -20,299503 &
  -40,316014 \\ \hline
M-F 1 &
  \begin{tabular}[c]{@{}l@{}}EMEF Arthur da \\ Costa e Silva\end{tabular} &
  \begin{tabular}[c]{@{}l@{}}Escola Municipal \\ de Ensino Fundamental\end{tabular} &
  \begin{tabular}[c]{@{}l@{}}Rua Presidente Rodrigues\\Alves, 255, República\end{tabular} &
  Vitória &
  -20,269288 &
  -40,294940 \\ \hline
M-F 10 &
  \begin{tabular}[c]{@{}l@{}}EMEF Ministro \\ Petrônio Portella\end{tabular} &
  \begin{tabular}[c]{@{}l@{}}Escola Municipal \\ de Ensino Fundamental\end{tabular} &
  Rua Serrana, 1 - Mata da Serra &
  Serra &
  -20,158285 &
  -40,247477 \\ \hline
M-F 11 &
  \begin{tabular}[c]{@{}l@{}}EMEF Governador \\ Carlos Lindenberg\end{tabular} &
  \begin{tabular}[c]{@{}l@{}}Escola Municipal \\ de Ensino Fundamental\end{tabular} &
  Rua Itabuna, 1 - Barro Branco &
  Serra &
  -20,163062 &
  -40,270502 \\ \hline
M-F 12 &
  EMEF São Marcos &
  \begin{tabular}[c]{@{}l@{}}Escola Municipal \\ de Ensino Fundamental\end{tabular} &
  \begin{tabular}[c]{@{}l@{}}Rua Maceió, s/n - São\\ Marcos II\end{tabular} &
  Serra &
  -20,184298 &
  -40,234423 \\ \hline
M-F 13 &
  EMEF Maria Anselmo &
  \begin{tabular}[c]{@{}l@{}}Escola Municipal \\ de Ensino Fundamental\end{tabular} &
  \begin{tabular}[c]{@{}l@{}}Rua Brasília, 262-308\\ - Alterosas \end{tabular}&
  Serra &
  -20,185134 &
  -40,232099 \\ \hline
M-F 14 &
  \begin{tabular}[c]{@{}l@{}}EMEF Dom José \\ Mauro Pereira Bastos\end{tabular} &
  \begin{tabular}[c]{@{}l@{}}Escola Municipal \\ de Ensino Fundamental\end{tabular} &
  \begin{tabular}[c]{@{}l@{}}Rua Arpoador, 1 - Morada \\ de Laranjeiras\end{tabular} &
  Serra &
  -20,191823 &
  -40,228955 \\ \hline
M-F 15 &
  \begin{tabular}[c]{@{}l@{}}EMEF Manoel \\ Carlos de Miranda\end{tabular} &
  \begin{tabular}[c]{@{}l@{}}Escola Municipal \\ de Ensino Fundamental\end{tabular} &
  \begin{tabular}[c]{@{}l@{}}Rua Angico, 12 - José de \\ Anchieta\end{tabular} &
  Serra &
  -20,193200 &
  -40,270784 \\ \hline
M-F 16 &
  \begin{tabular}[c]{@{}l@{}}EMEF Álvaro \\ de Castro Mattos\end{tabular} &
  \begin{tabular}[c]{@{}l@{}}Escola Municipal \\ de Ensino Fundamental\end{tabular} &
  \begin{tabular}[c]{@{}l@{}}Rua Brilhante, 88 - José \\ de Anchieta II\end{tabular} &
  Serra &
  -20,200713 &
  -40,277348 \\ \hline
M-F 17 &
  \begin{tabular}[c]{@{}l@{}}EMEF Professor \\ Luiz Baptista\end{tabular} &
  \begin{tabular}[c]{@{}l@{}}Escola Municipal \\ de Ensino Fundamental\end{tabular} &
  \begin{tabular}[c]{@{}l@{}}Rua 13 de Maio, 1 - Jardim \\ Tropical\end{tabular} &
  Serra &
  -20,205657 &
  -40,273192 \\ \hline
M-F 18 &
  EMEF Olivina Siqueira &
  \begin{tabular}[c]{@{}l@{}}Escola Municipal \\ de Ensino Fundamental\end{tabular} &
  \begin{tabular}[c]{@{}l@{}}Av. Dido Fontes, 1 - Jardim \\ Tropical\end{tabular} &
  Serra &
  -20,206789 &
  -40,274128 \\ \hline
M-F 19 &
  \begin{tabular}[c]{@{}l@{}}EMEF Dinorah \\ Pereira Barcelos\end{tabular} &
  \begin{tabular}[c]{@{}l@{}}Escola Municipal \\ de Ensino Fundamental\end{tabular} &
  \begin{tabular}[c]{@{}l@{}}Rua Assembleia de Deus, s/n - \\ Jardim Tropical\end{tabular} &
  Serra &
  -20,210045 &
  -40,272803 \\ \hline
M-F 2 &
  EMEF Augusto Ruschi &
  \begin{tabular}[c]{@{}l@{}}Escola Municipal \\ de Ensino Fundamental\end{tabular} &
  \begin{tabular}[c]{@{}l@{}}Rua Edson Pompemyaer, s/n,  \\ Manoel Plaza\end{tabular} &
  Serra &
  -20,232607 &
  -40,269669 \\ \hline
M-F 20 &
  \begin{tabular}[c]{@{}l@{}}EMEF Altair \\ Siqueira Costa\end{tabular} &
  \begin{tabular}[c]{@{}l@{}}Escola Municipal \\ de Ensino Fundamental\end{tabular} &
  \begin{tabular}[c]{@{}l@{}}Rua Nelcy Lopes Vieira, 21, \\ Jardim Limoeiro\end{tabular} &
  Serra &
  -20,218481 &
  -40,265727 \\ \hline
M-F 21 &
  \begin{tabular}[c]{@{}l@{}}EMEF Antônio \\ Vieira de Rezende\end{tabular} &
  \begin{tabular}[c]{@{}l@{}}Escola Municipal \\ de Ensino Fundamental\end{tabular} &
  \begin{tabular}[c]{@{}l@{}}Rua Domingos Martins, s/n - \\ Central Carapina\end{tabular} &
  Serra &
  -20,218590 &
  -40,272784 \\ \hline
M-F 22 &
  \begin{tabular}[c]{@{}l@{}}EMEF Aureníria \\ Correa Pimentel\end{tabular} &
  \begin{tabular}[c]{@{}l@{}}Escola Municipal \\ de Ensino Fundamental\end{tabular} &
  \begin{tabular}[c]{@{}l@{}}Rua Inhambú, s/n - Novo \\ Horizonte\end{tabular} &
  Serra &
  -20,218776 &
  -40,249304 \\ \hline
M-F 23 &
  \begin{tabular}[c]{@{}l@{}}EMEF Lacy \\ Zuleica Nunes\end{tabular} &
  \begin{tabular}[c]{@{}l@{}}Escola Municipal \\ de Ensino Fundamental\end{tabular} &
  \begin{tabular}[c]{@{}l@{}}Avenida Dois, 239 - Carapina \\ Grande\end{tabular} &
  Serra &
  -20,219149 &
  -40,280894 \\ \hline
M-F 24 &
  EMEF Novo Horizonte &
  \begin{tabular}[c]{@{}l@{}}Escola Municipal \\ de Ensino Fundamental\end{tabular} &
  \begin{tabular}[c]{@{}l@{}}Rua Sanhaço, s/n - Novo \\ Horizonte\end{tabular} &
  Serra &
  -20,220076 &
  -40,247734 \\ \hline
M-F 25 &
  EMEF Elpídia Coimbra &
  \begin{tabular}[c]{@{}l@{}}Escola Municipal \\ de Ensino Fundamental\end{tabular} &
  \begin{tabular}[c]{@{}l@{}}Rua Granada, s/n - André\\ Carloni\end{tabular} &
  Serra &
  -20,226853 &
  -40,283708 \\ \hline
M-F 26 &
  \begin{tabular}[c]{@{}l@{}}EMEF Américo \\ Guimarães Costa\end{tabular} &
  \begin{tabular}[c]{@{}l@{}}Escola Municipal \\ de Ensino Fundamental\end{tabular} &
  \begin{tabular}[c]{@{}l@{}}Rua Constante Nery, s/n - \\ Carapina Grande\end{tabular} &
  Serra &
  -20,228316 &
  -40,277129 \\ \hline
M-F 27 &
  EMEF Padre Gabriel &
  \begin{tabular}[c]{@{}l@{}}Escola Municipal \\ de Ensino Fundamental\end{tabular} &
  \begin{tabular}[c]{@{}l@{}}Rua Castelo, 27-145 - Jardim \\ Carapina\end{tabular} &
  Serra &
  -20,232150 &
  -40,284558 \\ \hline
M-F 28 &
  EMEF João Paulo II &
  \begin{tabular}[c]{@{}l@{}}Escola Municipal \\ de Ensino Fundamental\end{tabular} &
  \begin{tabular}[c]{@{}l@{}}Rua Jerônimo Monteiro, 260 -\\  Jardim Carapina\end{tabular} &
  Serra &
  -20,234482 &
  -40,288445 \\ \hline
M-F 29 &
  \begin{tabular}[c]{@{}l@{}}EMEF Espaço Alternativo\\Jardim Carapina\end{tabular} &
  \begin{tabular}[c]{@{}l@{}}Escola Municipal \\ de Ensino Fundamental\end{tabular} &
  \begin{tabular}[c]{@{}l@{}}Rua Mariano Souza Ramos, \\s/n - Boa Vista II\end{tabular} &
  Serra &
  -20,236998 &
  -40,281461 \\ \hline
M-F 3 &
  \begin{tabular}[c]{@{}l@{}}EMEF Izaura \\ Marques da Silva\end{tabular} &
  \begin{tabular}[c]{@{}l@{}}Escola Municipal \\ de Ensino Fundamental\end{tabular} &
  \begin{tabular}[c]{@{}l@{}}Av. Leitão da Silva, 3291, \\ Andorinhas\end{tabular} &
  Vitória &
  -20,286174 &
  -40,306467 \\ \hline
M-F 30 &
  EMEF Clotilde Rato &
  \begin{tabular}[c]{@{}l@{}}Escola Municipal \\ de Ensino Fundamental\end{tabular} &
  \begin{tabular}[c]{@{}l@{}}Rua Rui Barbosa, s/n - \\ De Fátima\end{tabular} &
  Vitória &
  -20,247344 &
  -40,264425 \\ \hline
M-F 31 &
  \begin{tabular}[c]{@{}l@{}}EMEF Adevalni Sysesmundo\\ Ferreira de Azevedo\end{tabular} &
  \begin{tabular}[c]{@{}l@{}}Escola Municipal \\ de Ensino Fundamental\end{tabular} &
  \begin{tabular}[c]{@{}l@{}}Rua Victorino Cardoso, 140 - \\ Jardim Camburi\end{tabular} &
  Vitória &
  -20,251440 &
  -40,268526 \\ \hline
M-F 32 &
  \begin{tabular}[c]{@{}l@{}}EMEF Elzira \\ Vivacqua dos Santos\end{tabular} &
  \begin{tabular}[c]{@{}l@{}}Escola Municipal \\ de Ensino Fundamental\end{tabular} &
  \begin{tabular}[c]{@{}l@{}}Rua Italina Pereira Mota, 501 - \\ Jardim Camburi\end{tabular} &
  Vitória &
  -20,256924 &
  -40,267377 \\ \hline
M-F 33 &
  \begin{tabular}[c]{@{}l@{}}EMEF Juscelino \\ Kubitschek de Oivelira\end{tabular} &
  \begin{tabular}[c]{@{}l@{}}Escola Municipal \\ de Ensino Fundamental\end{tabular} &
  \begin{tabular}[c]{@{}l@{}}Rua Jerônimo Vervloet, 880 -\\  Maria Ortiz\end{tabular} &
  Vitória &
  -20,257576 &
  -40,299133 \\ \hline
M-F 34 &
  \begin{tabular}[c]{@{}l@{}}EMEF Marechal \\ Mascarenhas de Moraes\end{tabular} &
  \begin{tabular}[c]{@{}l@{}}Escola Municipal \\ de Ensino Fundamental\end{tabular} &
  \begin{tabular}[c]{@{}l@{}}Rua Jerônimo Vervloet, 560 - \\ Maria Ortiz\end{tabular} &
  Vitória &
  -20,260419 &
  -40,298852 \\ \hline
M-F 35 &
  \begin{tabular}[c]{@{}l@{}}EMEF Maria Madalena \\ de Oliveira Domingues\end{tabular} &
  \begin{tabular}[c]{@{}l@{}}Escola Municipal \\ de Ensino Fundamental\end{tabular} &
  \begin{tabular}[c]{@{}l@{}}Rua Carlos Delgado Guerra \\Pinto, 450 - Jardim Camburi\end{tabular} &
  Vitória &
  -20,261826 &
  -40,270172 \\ \hline
M-F 36 &
  EMEF Adão Benezath &
  \begin{tabular}[c]{@{}l@{}}Escola Municipal \\ de Ensino Fundamental\end{tabular} &
  \begin{tabular}[c]{@{}l@{}}Rua Profª Clara Lima, 63 - \\ Antônio Honório\end{tabular} &
  Vitória &
  -20,263314 &
  -40,298224 \\ \hline
M-F 37 &
  \begin{tabular}[c]{@{}l@{}}EMEF Éber Louzada \\ Zippinotti\end{tabular} &
  \begin{tabular}[c]{@{}l@{}}Escola Municipal \\ de Ensino Fundamental\end{tabular} &
  \begin{tabular}[c]{@{}l@{}}Rua Natalina Daher Carneiro,\\ 815 - Jardim da Penha\end{tabular} &
  Vitória &
  -20,281454 &
  -40,296573 \\ \hline
M-F 38 &
  \begin{tabular}[c]{@{}l@{}}EMEF Professor Vercenílio \\ da Silva Pascoal\end{tabular} &
  \begin{tabular}[c]{@{}l@{}}Escola Municipal \\ de Ensino Fundamental\end{tabular} &
  \begin{tabular}[c]{@{}l@{}}Rua José Martins Delazari, 200 \\ - Joana D'arc\end{tabular} &
  Vitória &
  -20,283804 &
  -40,312921 \\ \hline
M-F 39 &
  \begin{tabular}[c]{@{}l@{}}EMEF Álvaro \\ de Castro Mattos\end{tabular} &
  \begin{tabular}[c]{@{}l@{}}Escola Municipal \\ de Ensino Fundamental\end{tabular} &
  \begin{tabular}[c]{@{}l@{}}Rua Eugenílio Ramos, 781 \\ - Jardim da Penha\end{tabular} &
  Vitória &
  -20,286543 &
  -40,298527 \\ \hline
M-F 4 &
  EMEF São Diogo &
  \begin{tabular}[c]{@{}l@{}}Escola Municipal \\ de Ensino Fundamental\end{tabular} &
  \begin{tabular}[c]{@{}l@{}}Rua Olavo Bilac, s/n, \\ São Diogo II\end{tabular} &
  Serra &
  -20,218284 &
  -40,257755 \\ \hline
M-F 40 &
  \begin{tabular}[c]{@{}l@{}}EMEF Orlandina \\ D Almeida Lucas\end{tabular} &
  \begin{tabular}[c]{@{}l@{}}Escola Municipal \\ de Ensino Fundamental\end{tabular} &
  \begin{tabular}[c]{@{}l@{}}Rua Luiz Gomes Tavares, 95 \\ - São Cristóvão\end{tabular} &
  Vitória &
  -20,289666 &
  -40,314410 \\ \hline
M-F 41 &
  \begin{tabular}[c]{@{}l@{}}EMEF Ceciliano \\ Abel de Almeida\end{tabular} &
  \begin{tabular}[c]{@{}l@{}}Escola Municipal \\ de Ensino Fundamental\end{tabular} &
  \begin{tabular}[c]{@{}l@{}}Rua Dr. Arlindo Sodré, 416 \\ - Itararé\end{tabular} &
  Vitória &
  -20,294313 &
  -40,307133 \\ \hline
M-F 42 &
  EMEF Octacílio Lomba &
  \begin{tabular}[c]{@{}l@{}}Escola Municipal \\ de Ensino Fundamental\end{tabular} &
  \begin{tabular}[c]{@{}l@{}}Rua Adolfo Cassoli, 198 - \\ São Cristóvão\end{tabular} &
  Vitória &
  -20,294819 &
  -40,315976 \\ \hline
M-F 43 &
  EMEF Zilda Andrade &
  \begin{tabular}[c]{@{}l@{}}Escola Municipal \\ de Ensino Fundamental\end{tabular} &
  \begin{tabular}[c]{@{}l@{}}Av. Professor Hermínio \\Blackman, 778 - Bonfim\end{tabular} &
  Vitória &
  -20,298703 &
  -40,310543 \\ \hline
M-F 44 &
  EMEF Otto Ewald Júnior &
  \begin{tabular}[c]{@{}l@{}}Escola Municipal \\ de Ensino Fundamental\end{tabular} &
  \begin{tabular}[c]{@{}l@{}}Rua Daniel Abreu Machado, \\ 549 - Itararé\end{tabular} &
  Vitória &
  -20,299040 &
  -40,306549 \\ \hline

\end{longtable}
\end{landscape}




  \chapter{Título do anexo 2}
\label{apendiceB}
Conteúdo do Apêndice B:



\end{document}