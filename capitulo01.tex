\chapter{Introdução}

Apresentar uma descrição muito ampla e generalista do contexto no qual a pesquisa está inserida, o problema em questão e a motivação para realizar essa pesquisa

%% substituir por texto 
\lipsum[1]

\section{Objetivo}
\subsection{Objetivo Geral}

Apresentar CLARAMENTE o objetivo do trabalho, ou seja, O QUE está sendo feito. Os objetivos, geralmente, são verbos como analisar, desenvolver, comparar, propor, elaborar, identificar etc.

\subsection{Objetivo Específico}

Dentro desse objetivo geral, encontram-se como metas específicas:
\begin{itemize}
    \item Identificar algo; 
  
    \item Comparar alguma coisa;
   
   \item Fazer um diagnóstico ...;
   
   \item Verificar e analisar ...;
   
    \item Quantificar ...
\end{itemize}

\section{Metodologia}

Apresentar COMO o seu problema será ``resolvido", ou seja, quais são as etapas necessárias para atingir o(s) objetivo(s).

Não explique aqui O QUE é uma metologia ou os tipos de metologias que existem. 
Explique A SUA metodologia para alcançar o objetivo desejado.



\section{Estrutura do Trabalho}

Colocar aqui a estrutura sucintamente. Esse trabalho está organizado ... No capítulo 2 é apresentado isso . . Tal coisa é apresentada no cap. 3 ...

