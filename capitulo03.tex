\chapter{Título do capítulo 3}

Este capítulo, em geral, explica os procedimentos da metodologia.

\section{Breve histórico}

%substituir
\lipsum[1]

\section{Uso de Siglas e Abreviaturas no texto}
Para usar as siglas e abreviações definidas no arquivo abrev.txt, podem ser usadas de 3 formas:
Se quiser escrever ``Racícionio Baseado em Casos (RBC)'' de forma completa, ou seja, com nome completo e a sigla entre parênteses use o comando $\backslash$acrfull\{RBC\}:
Usando o \acrfull{RBC} é possível ...
Se desejar usar apenas a sigla escreva $\backslash$acrshort\{RBC\}:
O \acrshort{RBC} é muito útil ...

Se for usar apenas o texto, sem a sigla, utilize o comando $\backslash$acrlong\{RBC\}

O \acrlong{RBC} é muito usado em ...

Outro exemplo é o uso de \acrfull{IA}. 


Somente as siglas e abreviaturas cadastradas no arquivo abrev.txt \textbf{E UTILIZADAS NO TEXTO} serão inseridas na Lista de Siglas, com os respectivos links para o local onde a sigla foi usada pela primeira vez.

\section{Símbolos e Variáveis}

Os símbolos e variáveis usadas ao longo do texto podem ser incluídas em uma Lista de Símbolos na parte pré-textual.
Para isso, crie primeiro a lista de símbolos, conforme descrito no arquivo simbolos.tex. Em seguida use as definições no texto 

% \begin{equation} \label{eu_eqn}
% \begin{split}
% Lending_{it} = {\abscoef} + {\tempo}\
% \end{split}
% \end{equation}

\section{Quadros ...}

Somente no Brasil são usados ``Quadros'' na ABNT.
Veja como colocar um quadro: 
O Quadro~\ref{qd:Teste} apresenta um exemplo de um quadro no formato de tabela.
\begin{quadro}[h]
  \caption{Exemplo de um quadro formatado como tabela (o caption dos quadros vem antes do quadro)}
\begin{center}
  \begin{tabular}{|c|c|} \hline
     Informação A & Alguma coisa relativa \\ \hline
     Informação B & Outra coisa \\ \hline
  \end{tabular}
\end{center}
\label{qd:Teste}
\end{quadro}

Outro exemplo.

\begin{quadro}[h]
\caption{Legenda de um quadro qualquer}
\label{quad:primeiro_quadro}
\centering
\begin{tabular}{|lllll|}
\cline{1-5}
A& B & C& D &E \\ \cline{1-5}
\multirow{3}{*}{1} & 2 & 3& 4& 5 \\
& 2 & 3& 4& 5 \\
& 2 & 3& 4& 5 \\
\cline{1-5}
\end{tabular}
\end{quadro}

Mais um exemplo:

\begin{quadro}[htb]
\caption{\label{quadro_exemplo}Exemplo de quadro}
\begin{center}
\begin{tabular}{|c|c|c|c|}
	\hline
	\textbf{Pessoa} & \textbf{Idade} & \textbf{Peso} & \textbf{Altura} \\ \hline
	Marcos & 26    & 68   & 178    \\ \hline
	Ivone  & 22    & 57   & 162    \\ \hline
	...    & ...   & ...  & ...    \\ \hline
	Sueli  & 40    & 65   & 153    \\ \hline
\end{tabular}\par 
Fonte: o Autor.
\end{center}
\end{quadro}

Este parágrafo apresenta como referenciar o quadro no texto, requisito
obrigatório da ABNT. 
Primeira opção, utilizando \texttt{autoref}: Ver o \autoref{quadro_exemplo}. 
Segunda opção, utilizando  \texttt{ref}: Ver o Quadro \ref{quadro_exemplo}.